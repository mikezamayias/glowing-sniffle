\documentclass[14pt]{extarticle}
% \documentclass[12pt, fleqn, leqno]{article}
\usepackage{perpage}
%\documentclass[12pt]{extreport}
\usepackage[margin=2cm,includeheadfoot,a4paper]{geometry}
\usepackage[english,greek]{babel}
\usepackage{indentfirst}
\usepackage[dvipsnames]{xcolor}
\usepackage{titlesec}
\usepackage{amsmath, mathtools}
\usepackage{xifthen, xparse}
\usepackage{fancyhdr}
\usepackage{fancyvrb}
\usepackage[hyphens]{url}
% \usepackage{hyperref}

\MakePerPage{footnote} 

%\setlength{\mathindent}{0pt}
\setlength{\headheight}{17pt}

%\renewcommand{\arraystretch}{1.5}

\titleformat{\chapter}[display]
  {\normalfont\bfseries}{}{0pt}{\Huge}

\setlength{\parskip}{0cm}
\setlength{\parindent}{1cm}

\pagenumbering{arabic}

\pagestyle{fancy}
\fancyhf{}
\fancyhead[R]{\rightmark}
\lhead{Διάλεξη 00}
\chead{Εφαρμοσμένα Μαθηματικά}
\rhead{Διαφορικές Εξισώσεις}
\cfoot{\thepage}

\begin{document}

\subsection{Συνήθης διαφορική εξίσωση}

\begin{equation*}
    \alpha y''(x) + \beta (x)+ \gamma = 0
\end{equation*}

Είναι γραμμική, δεύτερης τάξης, ομογενής, με σταθερούς συντελεστές.


\subsection{Χαρακτηριστική εξίσωση}

\begin{equation*}
    \alpha r^2 + \beta r + \gamma = 0
\end{equation*}

Διακρίνονται οι παρακάτω περιπτώσεις:
\begin{enumerate}
    \item \begin{equation*}
              \Delta > 0 \implies r_1, r_2 \in \Re : y(x) = c_1 e^{r_1x} + c_2 e^{r_2x}
          \end{equation*}
    \item
    \item
\end{enumerate}



\end{document}