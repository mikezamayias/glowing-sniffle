\documentclass[]{book}
\usepackage[margin=2.1cm, includeheadfoot, a4paper]{geometry}
\usepackage{babel}
\usepackage[normalize-symbols, keep-semicolon]{alphabeta}
\usepackage[dvipsnames]{xcolor}
\usepackage{amsmath, mathtools}
\usepackage{perpage}
\usepackage{multicol}
\usepackage[hyphens]{url}
\usepackage[unicode]{hyperref}

\hypersetup{
    colorlinks=true,
    linkcolor=black,
    filecolor=magenta,
    urlcolor=cyan,
    pdfauthor=author
}

\MakePerPage{footnote}

\setlength{\parskip}{0cm}
\setlength{\parindent}{1.2cm}

\pagenumbering{arabic}

\begin{document}
\tableofcontents
\chapter{Συνήθεις Διαφορικές Εξισώσεις}
\begin{multicols*}{2}

\section{Θεωρία}
\subsection{Συνήθης Διαφορική Εξίσωση (Σ.Δ.Ε.)}
\begin{equation*}
    \begin{aligned}
        \alpha y''(x) + \beta y' (x)+ \gamma y = 0
    \end{aligned}
\end{equation*}
Η Σ.Δ.Ε. είναι γραμμική, δεύτερης τάξης, ομογενής, με $α, β, γ$ σταθερούς συντελεστές.
\subsection{Χαρακτηριστική Eξίσωση}
\begin{equation*}
    \begin{aligned}
        \alpha r^2 + \beta r + \gamma = 0
    \end{aligned}
\end{equation*}
Για την γενική λύση της Χ.Ε. διακρίνονται οι παρακάτω περιπτώσεις:
\begin{enumerate}
    \item $Δ>0$
            \begin{equation*}
                \begin{aligned}
                    r_1, r_2 & \in \Re                       \\
                    y(x)     & = c_1 e^{r_1x} + c_2 e^{r_2x} \\
                    y_1(x)   & = c_1 e^{r_1x}                \\
                    y_2(x)   & = c_2 e^{r_2x}                \\
                    c_1, c_2 & \text{ σταθερές}
                \end{aligned}
            \end{equation*}
    \item $Δ=0$
            \begin{equation*}
                \begin{aligned}
                    r        & \in \Re                     \\
                    y(x)     & = c_1 e^{rx} + c_2 x e^{rx} \\
                    y_1(x)   & = c_1 e^{rx}                \\
                    y_2(x)   & = c_2 x e^{rx}              \\
                    c_1, c_2 & \text{ σταθερές}
                \end{aligned}
            \end{equation*}
    \item $Δ<0$
            \begin{equation*}
                \begin{aligned}
                    r_1                      & = A + Bi                                    \\
                    r_2                      & = A - Bi                                    \\
                    A        = \frac{-β}{2α} & \text{, } B = \frac{\sqrt{-Δ}}{2α}          \\
                    y(x)                     & = c_1 e^{Ax} \sin{Bx} + c_2 e^{Ax} \cos{Bx} \\
                    y_3(x)                   & = c_1 e^{Ax} \sin{Bx}                       \\
                    y_4(x)                   & = c_2 e^{Ax} \cos{Bx}                       \\
                    c_1, c_2                 & \text{ σταθερές}
                \end{aligned}
            \end{equation*}
\end{enumerate}
Η απόδειξη περίπτωης \(Δ<0\)
Είναι:
\begin{equation}\label{eqn:1}
    y_1(x) = e^{Ax} (\sin{Bx} + i \cos{Bx})
\end{equation}
\begin{equation}\label{eqn:2}
    y_2(x) = e^{Ax} (\sin{Bx} - i \cos{Bx})
\end{equation}

Έχουμε:
\begin{equation*}
    \begin{aligned}
        \eqref{eqn:1} + \eqref{eqn:2}                                   \\
        e^{Ax} (\sin{Bx} + i \cos{Bx}) + e^{Ax} (\sin{Bx} - i \cos{Bx}) \\
        e^{Ax} (\sin{Bx} + i \cos{Bx} + \sin{Bx} - i \cos{Bx})          \\
        e^{Ax} (2 \sin{Bx})                                             \\
        \implies                                                        \\
        y_1(x) + y_2(x)                         = 2 e^{Ax} \sin{Bx}     \\
        \frac{1}{2} y_1(x) + \frac{1}{2} y_2(x) = e^{Ax} \sin{Bx}       \\
        y_3(x)                                  = e^{Ax} \sin{Bx}
    \end{aligned}
\end{equation*}
Και:
\begin{equation*}
    \begin{aligned}
        \eqref{eqn:1} - \eqref{eqn:2}                                     \\
        e^{Ax} (\sin{Bx} + i \cos{Bx}) - e^{Ax} (\sin{Bx} - i \cos{Bx})   \\
        e^{Ax} (\sin{Bx} + i \cos{Bx} - \sin{Bx} + i \cos{Bx})            \\
        e^{Ax} (2 i \cos{Bx})                                             \\ \implies                                      \\
        y_1(x) + y_2(x)                             = 2 i e^{Ax} \cos{Bx} \\
        \frac{1}{2 i} y_1(x) + \frac{1}{2 i} y_2(x) = e^{Ax} \cos{Bx}     \\
        y_4(x)                                      = e^{Ax} \cos{Bx}
    \end{aligned}
\end{equation*}
\section{Παραδείγματα}
\subsection{Εύρεση Γ.Λ. Σ.Δ.Ε.}
\begin{enumerate}
    \item \( y''  -y' - 6y = 0 \)
    \begin{equation*}
        \begin{aligned}
            r^2 - r - 6 = 0 \text{, } \Delta = 25 > 0                         \\
            r_1 = \frac{-(-1)+\sqrt{Δ}}{-2} = \frac{-(-1)+\sqrt{25}}{-2} = -3 \\
            r_2 = \frac{-(-1)-\sqrt{Δ}}{-2} = \frac{-(-1)-\sqrt{25}}{-2} = 2  \\
        \end{aligned}
    \end{equation*}
    Άρα η γενική λύση είναι η
    \begin{equation*}
        \begin{aligned}
            y(x) = c_1 e^{-3x} + c_2 e^{2x}
        \end{aligned}
    \end{equation*}
    \item \( y'' -4y' - 5y = 0 \)
    \begin{equation*}
        \begin{aligned}
            r^2 - 4r - 5 = 0 \text{, } \Delta = -4 < 0 \\
            r_1 = \frac{-(-4)+i\sqrt{-Δ}}{-2} = 2 + i  \\
            r_2 = 2 - i                                \\
        \end{aligned}
    \end{equation*}
    Άρα η γενική λύση είναι η
    \begin{equation*}
        \begin{aligned}
            y(x) = c_1 e^{2x} \sin{x} + c_2 e^{2x} \cos{}
        \end{aligned}
    \end{equation*}
    \item \( y'' -4y' - 4y = 0 \)
\begin{equation*}
    \begin{aligned}
        r^2 - r - 6 = 0 \text{, } \Delta = 25 > 0                         \\
        r_1 = \frac{-(-1)+\sqrt{Δ}}{-2} = \frac{-(-1)+\sqrt{25}}{-2} = -3 \\
        r_2 = \frac{-(-1)-\sqrt{Δ}}{-2} = \frac{-(-1)-\sqrt{25}}{-2} = 2  \\
    \end{aligned}
\end{equation*}
Άρα η γενική λύση είναι η
\begin{equation*}
    \begin{aligned}
        y(x) = c_1 e^{-3x} + c_2 e^{2x}
    \end{aligned}
\end{equation*}
\end{enumerate}
\subsection{Προβλήματα αρχικών (ή συνοριακών) τιμών Σ.Δ.Ε}
\begin{enumerate}
    \item $ y'' + 4y' = 0 \text{, } y(0) = 0 \text{, } y(\frac{\pi}{12}) = 1 $ \\\\
    Η Χ.Ε. είναι η $$ r^2 + 4 = 0 $$
    Οι ρίζες της Χ.Ε. είναι οι $$ r_1 = 2i, r_2 = -2i $$
    Η Γ.Λ. της Δ.Ε. δίνεται από
    \begin{equation*}
        \begin{aligned}
            y(x) & = c_1 e^{0x} \sin{2x} + c_2 e^{0x} \cos{2x} \\
            y(x) & = c_1 \sin{2x} + c_2 \cos{2x}
        \end{aligned}
    \end{equation*}
    \item $ y'' + 4y' = 0 \text{, } y(0) = 0 \text{, } y(\pi) = 1 $ \\\\
    \item $ y'' + 4y' = 0 \text{, } y(0) = 0 \text{, } y(\pi) = 0 $ \\\\
    \item $ y'' - 2y' + 2y = 0 \text{, } y(0) = 0 \text{, } y'(0) = 2 $ \\\\
\end{enumerate}
\subsection{Επίλυση Γ.Λ. των παρακάτω Σ.Δ.Ε.}
\begin{enumerate}
    \item $ y'' - 2y' - 3y = 0 $
    \item $ y'' - 4y' = 0 $
    \item $ y'' - 4y' + 4y = 0 $
\end{enumerate}
\end{multicols*}
\chapter{Μετασχηματισμός Laplace}
\chapter{Σειρές Fourier}
\chapter{Πιθανότητες}
\end{document}