\documentclass[14pt, fleqn, leqno]{extreport}
\usepackage{perpage}
%\documentclass[12pt]{extreport}
\usepackage[margin=2cm,includeheadfoot,a4paper]{geometry}
\usepackage{fontspec}
%\usepackage[utf8x]{inputenc}
\usepackage[english,greek]{babel}
\usepackage{indentfirst}
\usepackage[dvipsnames]{xcolor}
\usepackage{listings}
\usepackage{titlesec}
\usepackage{amsmath, mathtools}
\usepackage{xifthen, xparse}
\usepackage{fancyhdr}
\usepackage{fancyvrb}
\usepackage[autostyle,english=american]{csquotes}
\usepackage[hyphens]{url}
\usepackage{hyperref}

\MakePerPage{footnote} 

%\setlength{\mathindent}{0pt}
\setlength{\headheight}{17pt}

%\renewcommand{\arraystretch}{1.5}

\titleformat{\chapter}[display]
  {\normalfont\bfseries}{}{0pt}{\Huge}

\setlength{\parskip}{0cm}
\setlength{\parindent}{1cm}

\setmainfont{[EBGaramond-Regular.ttf]}
\setmonofont{[FiraMono-Regular.otf]}

\MakeOuterQuote{"}

\hypersetup{
    colorlinks = true,
    linkcolor=black,
    filecolor=magenta,
    urlcolor=blue,
    pdftitle={Project 2}
}


%\definecolor{name}{model}{color-spec}

\lstdefinestyle{mystyle}{
    language=C,
    backgroundcolor=\color{white},   
    commentstyle=\color{teal},
    keywordstyle=\color{blue},
    numberstyle=\color{gray}\ttfamily,
    stringstyle=\color{orange},
    basicstyle=\ttfamily\footnotesize,
    breakatwhitespace=false,         
    breaklines=true,                 
    captionpos=b,                    
    keepspaces=true,                 
    numbers=left,                    
    numbersep=5pt,                  
    showspaces=false,                
    showstringspaces=false,
    showtabs=false,                  
    tabsize=2,
    frame=lines,
    framesep=0.1cm,
    rulecolor=\color{black},
    morestring=[b]"
}

\lstset{style=mystyle}

\pagenumbering{arabic}

\pagestyle{fancy}
\fancyhf{}
\fancyhead[R]{\rightmark}
\lhead{Σημειώσεις εργαστηρίου}
\chead{Δομές Δεδομένων}
\cfoot{\thepage}

\newcommand\rowop[1]{\scriptstyle\smash{\xrightarrow[\vphantom{#1}]{\mkern-4mu#1\mkern-4mu}}}

\DeclareDocumentCommand\converttorows
{>{\SplitList{,}}m}
{\ProcessList{#1}{\converttorow}}
\NewDocumentCommand{\converttorow}{m}
{\ifthenelse{\isempty{#1}}{}{\rowop{#1}}\\}

\DeclareDocumentCommand \rowops{m}
{\;
 \begin{matrix}
\converttorows {#1}
 \end{matrix}
 \; }


\begin{document}

\title{Δομές Δεδομένων\\Σημειώσεις εργαστηρίου}
\author{Μιχαήλ Ανάργυρος Ζαμάγιας\\ΤΠ5000}
\date{\today}
\maketitle
\newpage

\tableofcontents

\newpage
\chapter{Εργαστήριο 1}

\section{Μονοδιάστατοι πίνακες}

Ισχύουν τα εξής:
\begin{lstlisting}
pin == &pin[0]
pin+k == &pin[k]
*pin == pin[0]
*(pin+k) == pin[k]
\end{lstlisting}

Η τιμή ενός δείκτη ισούται με τη διεύθυνση μνήμης του byte στο οποίο είναι τοποθετημένος ο δείκτης και εμφανίζεται στην οθόνη με την χρήση του προσδιοριστή \lstinline{%p}.

\section{Πίνακες ως ορίσματα συνάρτησης}

Για να "περάσω" σε μια συνάρτηση ως παράμετρο ένα πίνακα, περνάω ένα δείκτη στην αρχή του πίνακα και (αν χρειάζεται) το μέγεθος του πίνακα.

Στον ορισμό μιας συνάρτησης (έστω της \lstinline{parad}) οι παρακάτω συμβολισμοί έιναι ισοδύναμοι:
\begin{lstlisting}
void parad(int *pin)
void parad(int pin[])
\end{lstlisting}
Σε κάθε περίπτωση, το \lstinline{pin} είναι δείκτης σε ακέραιο.

\section{Ταξινόμηση πίνακα}

Υπάρχουν διάφοροι αλγόριθμοι ταξινόμησης ενός πίνακα. Αυτός που περιγράφεται εδώ είναι γνωστός ως "Ταξινόμηση με επιλογή".

Η συνάρτηση θα ξεκινά από το στοιχείο της πρώτης θέσης τοτ πίνακα, το \lstinline{list[0]}, με στόχο να τοποθετηθεί στην θέση αυτή η μικρότερη τιμή του πίνακα.Η συνάρτηση να διατρέχει όλα τα υπόλοιπα στοιχεία, από το \lstinline{list[1]} μέχρι το \lstinline{list[Ν-1]} και να συγκρίνει καθένα με το πρώτο. Αν βρει κάποιο μικρότερο από το πρώτο, τα στοιχεία εναλλάσσονται μεταξύ τους. Τελικά το \lstinline{list[0]} θα έχει την μικρότερη τιμή του πίνακα.

Αφού τελειώσουμε με το πρώτο στοιχείο επαναλαμβάνεται η διαδικασία, προσπαθώντας να βάλουμε στη θέση 1 του πίνακα τη δεύτερη σε μέγεθος τιμή. Συγκρίνονται δηλαδή όλα τα στοιχεία από το \lstinline{list[2]} και μεα με το \lstinline{list[1]}. Αν βρεθεί κάποιο στροιχείο μικρότερο από το \lstinline{list[1]}, τα στοιχε εναλλάσσονται μεταξύ τους, κ.ο.κ.

Χρειάζεστε δύο εμφωλευμένες επαναλήψεις.

\chapter{Εργαστήριο 2}

\section{Πίνακες συμβολοσειρών}

\chapter{Εργαστήριο 3}

\section{Δομές}
\section{Πίνακες δομών}
\section{Δομές ως παράμετροι και ως τιμή επιστδροφής συναρτήσεων}

\chapter{Εργαστήριο 4}

\section{Δυναμική δέσμευση μνήμης (συνάρτηση \lstinline{malloc})}
\section{Πίνακες δεικτών}

\chapter{Εργαστήριο 5}

\section{Στοίβες, υλοποίηση με πίνακα}

\chapter{Εργαστήριο 6}

\section{Απλά συνδεδεμένες λίστες (δημιουργία)}
\section{Λειτουργίες στις απλά συνδεδεμένες λίστες: αναζήτηση, εισαγωγή, διαγραφή, μετακίνηση, συνένωση λιστών}



\end{document}